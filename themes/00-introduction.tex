\section{Introduksjon}

Hei og velkommen! Dette er notatene som følger med den uoffisielle eksamensforelesning.
Først og fremst vill jeg si at jeg vet ingenting om hva som kommer på eksamen,
spesielt med tanke på at det nå blir hjemmeeksamen. Jeg er heller ikke en del av fagstaben
i faget, og gir ingen garantier i forhold til informasjonens korrekthet, og dette er utelukkende
et ekstra tilbud for å hjelpe dere med å komme igjennom dette tidvis utfordrende emnet.\\

\noindent Jeg har vært læringsassistent i et annet fag med noe overlapp, IMAT2021 - Matematiske metoder 2 for dataingeniør,
så jeg har erfaring med veiledning av studenter, og håper å benytte meg av dette i forbindelse med denne forelesningen.
Jeg har tidligere hatt både dette emnet, og TMA4140 - Diskret Matematikk, disse emnene er \textit{veldig} like, med 
få kapitler som skiller dem. Av min erfaring, så består dette faget i stor grad av mange nye definisjoner,
og en litt annen tilnærming til matematikk, enn det en er vant med fra tidligere fag, men oppgavene er sjeldent 
så veldig utfordrende, forutsagt at definisjonene sitter.\\

\noindent I løpet av forelesningen så ønsker jeg å gi en kjapp repetisjon av viktig teori,
men vil fort gå over til relevante eksamensoppgaver, og med utdypende fremgangsmåter, men husk at en ikke kan forvente
at oppgaver av en hvis type kommer på eksamen. Det er veldig vanlig at det kommer nye definisjoner som baserer seg på
kunnskap en skal ha lært, for så å løse enkle oppgaver basert på disse nye definisjonene. Det er derfor uhyre viktig
å være komfortabel med, og forstå den matematiske notasjonen som benyttes i faget, både for å forstå oppgavene, og
fordi faglæreren ofte er relativt streng i forbindelse med føring av fremgangsmåte. Dog vet jeg ikke hvordan den
digitale hjemmeeksamenen kommer til å gjennomføres. 
