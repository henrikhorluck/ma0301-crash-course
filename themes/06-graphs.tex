\section{Grafer}

\subsection{Terminologi}
Jeg vil gå igjennom det viktigste i forhold til hver oppgave, men dette er godt beskrevet i Wikipendium.
\subsection{Isomorfi}

Aka. er to grafer like?

\noindent Krav: Må kunne finne en \enquote{mapping} (funksjon) mellom alle noder og kanter.

Å teste ut alle mulige funksjoner tar \textit{mye} tid, siden det vokser fort.

Derfor ser man på enkelte \textit{egenskaper} ved grafer, siden disse må deles av begge grafene,
dersom de skal være like.

Da ser en ofte på egenskaper som graden til noder, og graden til nabo-noder, og sjekker at antallet der stemmer overrens.
Men det er også andre egenskaper å teste, som hvorvidt grafen er sammenhengene, og det kan en sjelden gang være lønnsomt å sjekke
den inverse grafen, siden den også skal være isomorf.

Homeomorfi, er en litt videre variant av isomorfi, hvor vi også tillatter elementær subdivisjoner, altså å splitte kanter


\subsection{Oppgaver}

\subsubsection{V2006 - Oppgave 6}
\textbf{a)} Er disse grafene isomorfe?
\begin{align*}
\begin{tikzpicture}
    \begin{scope}[every node/.style={shape=circle,draw, fill=black}]
        \node(1) {};
        \node(2)[right=of 1] {};
        \node(3)[below =of 1] {};
        \node(4)[below right = of 1] {};
    \end{scope}
    %Lines
    \draw[thick]   (1) -- (3)
            (3) -- (2)
            (3) -- (4); 
\end{tikzpicture}
\qquad
\qquad
\begin{tikzpicture}
    \begin{scope}[every node/.style={shape=circle,draw, fill=black}]
        \node(1) {};
        \node(2)[right=of 1] {};
        \node(3)[below =of 1] {};
        \node(4)[below right = of 1] {};
    \end{scope}
    %Lines
    \draw[thick]   (1) -- (3)
            (4) -- (2)
            (3) -- (4); 
\end{tikzpicture}
\end{align*}

\noindent \textbf{b)} Finn 11 løkke-frie(loop-free) ikke-isomorfe uretta(non-directed) grafer med 4 hjørner.\\
\textbf{c)} Vis at de 11 grafene du fant i b) er alle mulige løkke-frie ikke-isomorfe uretta grafer med 4 hjørner.\\
\indent Tips: tell kanter.

\paragraph*{Løsningsforslag} %TODO