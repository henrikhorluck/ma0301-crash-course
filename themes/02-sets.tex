\section{Mengdelære}

Mengder er en samling med \textit{distinkte} elementer. De har stort sett de samme reglene
som boolsk logikk, og \href{https://www.wikipendium.no/MA0301_Elementary_Discrete_Mathematics#sets}{Wikipendium} beskriver dette godt.
Det er viktig å ha god kontroll på dette, grunnet mengdelæren er en byggesten for mange av de senere temaene.\\
Ting å kunne til eksamen:
\begin{itemize}
    \item Medlem i mengden: \(x\in A\), \(x\not\in B\)
    \item Delmengder: \(A \subset B, A\subseteq B, A\not\subset B, A \not\subseteq B\)
    \item Vise likhet: \(A=B \Leftrightarrow A\subseteq B \land B \subseteq A\)
    \item Kardinalitet (antall elementer i en mengde): \(|A|\)
    \item Den tomme mengde, \(\emptyset = \{\} \left(\neq \{\{\}\} = \{\emptyset\}\right)\)
    \item Potensmengde (Power set): \(\mathcal{P}(A) = \{a ~|~ a \subseteq A\}\)
    \item Kardinaliteten til potensmengden: \(|\mathcal{P}(A)| = 2^{|A|}\)\\
    En må også kunne begrunne dette.
    \item Union, snitt, symmetrisk differanse, gjerne ved venn-diagram: \(\cup, \cap, \triangle\)
    \item Regneregler (som er identisk med logikken)
\end{itemize}

% Skriv om setbuilder-notasjon
\noindent Viktig å få med seg at mengdene \(\{a, a, a, a\}\) og \(\{a\}\) er identiske. Samme for \(\{a, b, a\}\) og \(\{b, a\}\)

\subsection{Oppgaver}

\subsubsection{V2008 - Oppgave 3}
La $A$ og $B$ være to mengder. Skriv mengden
\[
(A\cup B)-(B-A)    
\]
på kortest mulig måte.

\paragraph*{Løsningsforslag}
Anbefaler å teste ut med venn-diagram først, for å få en intuisjon for forventet resultat.
\begin{align*}
    (A\cup B)-(B-A) &= (A\cup B)\cap \overline{(B\cap\overline{A})}\\
    &= (A\cup B)\cap (\overline{B}\cup A)\\
    &= A\cup (B\cap \overline{B})\\
    &= A \cup \emptyset\\
    &= A
\end{align*}

\subsubsection{TMA4140 - K2019 Oppgave 1}

Avgjør om følgende likhet av mendge holder for alle mendger $X$ med
delmendger $A,B,C,D \subseteq X$.
\[
\overline{A \cup ((\overline{A} \cup B)\cap (C\cup \overline{B})\cap (D\cup \overline{C}))} \cup (\overline{A} \cup D) = X   
\]

\paragraph*{Løsningsforslag} % TODO