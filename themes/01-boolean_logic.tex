\section{Boolsk logikk}

Dette forklares veldig godt i \href{https://www.wikipendium.no/MA0301_Elementary_Discrete_Mathematics#logic}{Wikipendium}.
Jeg har lagt ved noen praktiske sannhetstabeller i Figur \ref{figure:truth_tables}, som må pugges.\\

\noindent Oppgaver består vanligvis av en av følgende:
\begin{enumerate}
    \item \textbf{Forenkle uttrykk} Da gjelder det å kunne \textit{Laws of Logic}, om du blir
    bedt om negasjon av et utrykk er det nøyaktig samme oppgave.
    OBS du kan \textit{ikke} benytte deg av implikasjons-reglene (Rules of inference),
    siden det er implikasjoner, og ikke ekvivalenser, altså er uttrykkene du kan utlede ikke 
    de samme som det du hadde i utgangspunkt. 
    \item \textbf{Vise at noe er en tautologi, eller motsigelse} Det løses ofte ved hjelp av 
    sannhetstabeller, men om de skulle bli for store, f.eks. grunnet $\geq 3$ påstander, så kan en 
    benytte seg av implikasjons-reglene, og ha fokus på de relevante variablene. Dersom du skal undersøke en implikasjon,
    og venstre side er stor, men høyre side er mye mindre, kan en fokusere på de verdiene av 
    variablene som fører til at høyre side er usann, og forsøke å få venstre side sann. Om du finner et sett med
    verdier for variablene, hvor venstre side er sann, men høyre er usann, så har du vist at det er en motsigelse.
    \item \textbf{Bekrefte validitet av konklusjon} Da får du vanligvis oppgitt en rekke påstander, som alle antas å stemme,
    og du så benytte deg av implikasjoner for å finne ut om den oppgitte konklusjonen er sann eller ikke.
\end{enumerate}

\begin{figure}[h]
\centering
\begin{align*}
    \begin{array}{|c|c|}
        \hline
        p & \neg p\\
        \hline
        1 & 0 \\
        0 & 1 \\
        \hline
    \end{array}
    \qquad
    \begin{array}{|c c|c|}
        \hline
        p & q & p \lor q\\
        \hline
        1 & 1 & 1\\
        1 & 0 & 1\\
        0 & 1 & 1\\
        0 & 0 & 0\\
        \hline
    \end{array}
    \qquad
    \begin{array}{|c c|c|}
        \hline
        p & q & p \land q\\
        \hline
        1 & 1 & 1\\
        1 & 0 & 0\\
        0 & 1 & 0\\
        0 & 0 & 0\\
        \hline
    \end{array}\\
    \begin{array}{|c c|c|}
        \hline
        p & q & p \veebar q\\
        \hline
        1 & 1 & 0\\
        1 & 0 & 1\\
        0 & 1 & 1\\
        0 & 0 & 0\\
        \hline
    \end{array}
    \qquad
    \begin{array}{|c c|c|}
        \hline
        p & q & p \to q\\
        \hline
        1 & 1 & 1\\
        1 & 0 & 0\\
        0 & 1 & 1\\
        0 & 0 & 1\\
        \hline
    \end{array}
    \qquad
    \begin{array}{|c c|c|}
        \hline
        p & q & p \leftrightarrow q\\
        \hline
        1 & 1 & 1\\
        1 & 0 & 0\\
        0 & 1 & 0\\
        0 & 0 & 1\\
        \hline
    \end{array}
\end{align*}
\caption{Sannhetstabellen til operatoren negasjon, logisk eller, logisk og, logisk eksklusiv eller, implikasjon, og ekvivalens}
\label{figure:truth_tables}
\end{figure}

\begin{figure}[h]
    \centering
    Exercise 8, Exercise set 1, 2019: Negate $(p \lor \neg q) \land \neg p$
        
    \begin{tabular}{ l | c c }
    \multicolumn{1}{l}{}
 &  \multicolumn{1}{c}{Step}
 & \multicolumn{1}{c}{Reasoning} \\
      \hline
    1 & $(p \lor \neg q) \land \neg p$ & Initial statement \\ 
    2 & $\neg ((p \lor \neg q) \land \neg p)$ & Step (1), Negation \\  
    3 & $\neg (p \lor q) \lor \neg ( \neg p)$ & Step (2), De Morgan's Law \\
    4 & $(\neg p \land \neg ( \neg q)) \lor p$ & Step (3), De Morgan's Law and Double negation\\
    5 & $(\neg p \land q) \lor p$ & Step (4), Double Negation\\
    6 & $p \lor (\neg p \land q)$ & Step (5), Associative law\\
    7 & $(p \lor \neg p) \land (p \lor q)$ & Step (6), Distributive law and Commutative law\\
    8 & $T_{0} \land (p \lor q)$ & Step (7), Inverse law\\
    9 & $p \lor q$ & Step (8), Identity law\\
    \hline
    $\therefore$ & $p \lor q$ & Final negated statement
    \end{tabular}
    \caption{En løsning på en oppgave med negasjon og forenkling}
    \label{}
\end{figure}
