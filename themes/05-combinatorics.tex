\section{Kombinatorikk}

\noindent Formler finnes i f.eks. Rottman's Matematiske Formelsamling, generelt ikke veldig intuitivt (sitat statistikk-foreleser)
prøv ut oppgaver, og jobb med å gjengjenne situasjoner som typ. summering av $x$-antall variabler til et bestemt tall.

\subsection{Oppgaver}
\subsubsection{TMA4140 - K2019, Oppgave 2a}
Hva er koeffisienten til leddet $x^4 y^6$ i polynomet $(2x-3y)^{10}$?

\paragraph*{Løsningsforslag}

Benytter Binomalteoremet
\[
(x+y)^n=\sum_{k=0}^{n}\binom{n}{k} x^k y^{n-k}
\]
Vi ønsker å se på ledd 4 i summen, der hvor $x$ har potens 4.
\begin{align}
    \binom{10}{4}(2x)^4(-3y)^6 &= \binom{10}{4}\cdot 2^4 \cdot x^4 \cdot (-3)^6 \cdot y^6\\
    &= \binom{10}{4}\cdot 2^4 \cdot (-3)^6 \cdot x^4 y^6\\
    &= \binom{10}{4} 2^4 3^6 \cdot x^4 y^6
\end{align}
Dermed har vi at koeffisienten blir $\binom{10}{4} 2^4 3^6 $

\subsubsection{TMA4140 - K2019, Oppgave 2b}
Hvor mange injektive funksjoner finnes det fra $\{1,2,3,4\}$ til $\{1,2,3,4,5,6,7,8\}$

\paragraph*{Løsningsforslag} 

Her må en huske definisjonen av en funksjon. Da har man for hvert element i definisjonsmengden/domenet 
ett valg i verdiområdet/kodomenet, som sammen utgjør bildet til funksjonen. Siden funksjonene skal være
injektive, blir det utvalg uten tilbakelegging, og siden det er fire elementer i domenet, og funksjonen
må ha et definert output for hvert element i domenet, må vi velge fire ganger. Dermed blir svaret:
\[
8\cdot 7 \cdot 6\cdot 5   
\]